\documentclass{article}
\usepackage{amsmath,amsfonts}
\usepackage{graphicx}
\usepackage[margin=1.0in,letterpaper]{geometry}
\usepackage{float}
\usepackage{hyperref}
%\usepackage{subfigure}

% are used to denote commments, everything after it on this line is a comment and while not be processed 

\begin{document}

\title{Basic Interferometry}
\author{Leonardo Sattler Cassara, e-mail:leosattler@berkeley.edu}
\date{April 7, 2014}
\maketitle

\begin{abstract}

The idea in this lab is to understand the working of an
interferometer. We use two antennas with a baseline separation of $10 /
meters$, operating with a spectral resolution of about $11 / GHz$. With
this interferometer, using $Python$ codes to remotely control this pair,
we point to three kinds of objects: the sun, the moon and a point
source. Exploring the least square fitting procedure, we collect data for
these three sources and find quantities related to the observation. For
three different point sources (Virgo A, Orion Nebula and M17) we calculate
the baseline separation. For the Sun and the Moon, we evaluate their
radius. 

\end{abstract}

\section{Introduction}

In astrometry, terferometry is known as the practice of using two or
more radio telescopes. It is a powerful method, since based on the
geometry of the equipments and the distance between them, we can achieve
a good frequency resolution. In this lab, we use our interferometer
configuration and perform the data analysis, recovering some of these
informations and finding some informations about the geometry of our sources.

In section 2, we introduce some of the quantities related to
interferometry that are largelly explored in this experiment, introduce
relevant ideas and
understand the working of this instrument. 

In section 3, via the \emph{PyEphem} module, we develop codes for the rotation
matrices, used to manipulate the coordinates of our targets. Also, we
outline the basic procedure to track an object and store data during our
observations, done in \emph{Python} using functions from the provided
\emph{radiolab} module.

In the remaining sections, we analyse the collected data, present the
general procedures and later discuss the results. 

\section{The Interferometer}

Our interferometer is a pair of antennas separated by a baseline $B = 10
/meters$ working at a about $\nu = 10.7 / GHz$. By definition, the spectral
resolution of this instrument is given by

\begin{equation}
R = \frac{\lambda}{B},
\label{eq:spec_res}
\end{equation} 

also known as fringe spacing. Since our antennas are spaced by a vector $\vec{B}$ on
the east-west direction, as the source moves in the sky they receive
different signal (amount of photons) during the time of
observation. It defines the \emph{geometric delay}, $\tau_{g}$, defined as:
\begin{equation}
\tau_{g} = \frac{\vec{B} \cdot \hat{s}}{c},
\label{eq:tau}
\end{equation}  
where $\hat{s}$ is a unit vector representing the distance from one of
the antennas, so each of them has a $\tau_{g}$. These quantities are
labelled in Fig.1. 

What this interferometer does is to multiply the two
signals using a mixer (a couple of mixers and amplifiers, in fact),
giving a result that is proportional to the flux of the source. This is
done by correlating the signals, say $E_{1}$ and $E_{2}$, that are
intrinsically proportional to the time of observation (hence, $tau$):
\begin{equation}
E(t)_{1} = cos(2 \pi \nu t), \text{ } E(t)_{2} = cos(2 \pi \nu t + \tau_{tot}),
\label{eq:ampl}
\end{equation} 
where $\tau_{tot} = \tau_{g} + \tau_{c}$ takes into account a constant delay
$\tau_{c}$ given by the cable length.

To take into account the unit vector to the source, we write our
geometrical delay in terms of the position of the source in the sky, its
hour angle $h_{s}$:
\begin{equation}
\tau_{g}(h_{s})=\left[\frac{B}{c}cos\delta \right]sinh_{s},
\label{eq:tau_ha}
\end{equation} 
and $\delta$ is the source's declination, which is fixed for sources as
Virgo A, Orion and M 17, but varies for the Sun and the Moon on a day. 

One thing about correlation: it ends up zeroing signals that are noise,
so our result is dependent of the brightness of our sources and not from
the sky. So the product of our interferometer is $E_{1} \times E_{2}$:
\begin{equation}
F(t) = cos(2 \pi \nu t)cos(2 \pi \nu t[t+ \tau_{tot}]),
\label{eq:imp}
\end{equation} 
what, in term of the hour angle $h_{s}$ is:
\begin{equation}
F(t) = cos(2 \pi \nu[\tau_{g}(h_{s}) + \tau_{c}]).
\label{eq:very_imp}
\end{equation} 
This previous equation is very important. It describes the \emph{fringe
  pattern}, a quantity that is really important to analyse any data from
the interferometer. 

\begin{figure}[H]
\center
\includegraphics[scale=0.42]{interf.jpg}
\caption {Geometrical description of our interferometer. Credits:
  Radiolab \emph{casper} website.} 
\label{Fig:1}
\end{figure}

\section{Using Our Instrument} 

The interferometer is controlled by a code using the \emph{radiolab}
module, which tracks the source in specified intervals and also 
performs a homing (gets back to rest position)  when set to. During data aquisition, our antennas will be following 
the objects as it rotates with the
Celestial Sphere. To describe an object's
position in the sky, it all depends on the \emph{Coordinate System} one
uses. Our antennas track given values in the \emph{Horizontal Coordinate
System} (\emph{Azimuth} and \emph{Altitude}), while the objects
positions are initially given in the \emph{Equatorial Coordinate System} (\emph{Right Ascention} and \emph{Declination}). So, first of all, we build
mathematical tools that will perform a change of coordinates for our
tracking procedure.

\subsection{Rotation Matrices} 

Although astronomical objects pass by in the sky as the Earth rotates,
if you look at a catalog
we have some of these objects specified at fixed points in the sky. As
said previously, it all depends on the Coordinate System you are looking
at, since each one of them have different characteristics (most
important, the origin). 

As said before, our antennas will track Azimuth (\emph{az}) and
Altitude (\emph{alt}) values of the object, but only after specified
their Right Ascentions (\emph{ra}) and Declinations (\emph{dec}), which
we get from catalogs. But, for the Sun and the moon, since
these values will vary within a day, we enter new values in
the Equatorial System each time we evaluate their value in the
Horizontal System. Reason: the origin of the Equatorial System is the
point where the Sun cross the Equator at the Spring's Equinox, known as
\emph{Vernal Point}. Since the
Sun change its position relative to this point
during the year (as goes through its path in the sky, the Eclipt), 
it will have relevant variations in a
day. This effect is even greater for the Moon, due to its proximity to
the Earth. For distant sources as stars and other galaxies, the variation of
this point would be accounted due to the Earth's spin variation, the
Precession, that ends up moving the origin of this system since the
Equator changes in relation to the Eclipt. But the Precession is
negligible in a day, so only the Sun and Moon will have different ra's
and dec's at each measurement of the interferometer.

To perform the change of coordinates, we will make a two-step transformation described by
the following equation:
\begin{equation}
\mathbf{R}_{(\alpha, \delta) \rightarrow (az, alt)}=\mathbf{R}^{2}_{(ha, \delta) \rightarrow
  (az, alt)} \cdot \mathbf{R}^{1}_{(\alpha, \delta)
  \rightarrow (ha, \delta)},
\label{eq:conv}
\end{equation}
where we are first applying $R^{1}$ that does not change the Coordinate
System, but simply rerwites the Azimuth of the object into Hour Angle
via the simple relation $ha = lst - \alpha$. $lst$ is our \emph{Local Sidereal
Time}, calculated via the \emph{PyEphem} and the \emph{Time} modules in
\emph{Python}. Our $R^{1}$ matrix is then defined as

\begin{equation}
\mathbf{R^{1}} =
 \begin{bmatrix}
  cos(lst) & sin(lst) & 0 \\
  sin(lst) & -cos(lst) & 0 \\  
  0 & 0 & 1
 \end{bmatrix}.
\end{equation}

At second, we apply $R^{2}$, and since $(ha, \delta)$ are Earth-based
coordinates, this matrix depends only on our terrestrial latitude
$\phi$: 

\begin{equation}
\mathbf{R^{2}} =
 \begin{bmatrix}
  -sin(\phi) & 0 & cos(\phi) \\
  0 & -1 & 0 \\  
  cos(\phi) & 0 & sin(\phi)
 \end{bmatrix},
\end{equation}
$\phi$ being $37.87 \ degrees$.

So $R$ is comprised of these two matrices, that we use to solve the
equation $\vec{r'}_{(az,alt)}=\textbf{R} \times
\vec{r}_{(\alpha,\delta)}$. To build our $\vec{r}$ vector, we use our
values of $(\alpha,\delta)$:

\begin{equation}
\vec{r} =
 \begin{bmatrix}
  cos(\delta) \times cos(\alpha) \\
  cos(\delta) \times sin(\alpha)  \\  
  sin(\delta) 
 \end{bmatrix},
\end{equation}
in order to work with the angles in rectangular coordinates and apply
the rotation.

The output of Eq.(7) applied to Eq.(10), $\vec{r'}_{(az,alt)}=\textbf{R} \times
\vec{r}_{(\alpha,\delta)}$, will then be a $3 \times 1$ matrix,
$\vec{r'}_{3 \times 1}$, and our $(az,alt)$ will be evaluated as
follows:

\begin{equation}
\begin{gathered}
az = \arctan{\left(\frac{\vec{r'}_{2,1}}{\vec{r'}_{1,1}}\right)}, \\
alt = \arcsin{(\vec{r'}_{3,1})}.
\end{gathered}
\label{eq:3_f}
\end{equation} 

The codes that built the rotation matrices and performed the
Coordinate System
transformation, that were used through the data aquisition for the analyzed objects, can be found at my \emph{Github} account \href{https://github.com/leosattler/ugradio/tree/master/lab_interf}
{here}.

\subsection{The Controlling Codes} 

The values of Eq.(\ref{eq:3_f}) are the input of our codes to control
the antennas. For so, together with the support of the \emph{Time}
module, we imported the \emph{radiolab} module and developed a code that
performed the data aquisition for this lab. Using the
function \emph{pntTo}, which takes the arguments (alt,az), we pointed
the antennas to a desired position in the sky. Since the
Earth rotates at a speed of $15$ degrees per hour, we call this function
every $30 \ seconds$ to keep track of our objects. Besides that, we perform a homing every $1 \
hour$ because, after around $100$ pointing commands, the interferometer starts to
loose precision on doing this task. 

This all is done via a \emph{thread} in our codes. Another \emph{thread}
is set to run to record the data via the function \emph{recordDVM}. This
function stores the data in \emph{.npz} files with the following information: the Right Ascention ($ra$)
of the object in decimal hours, the Declination ($dec$) in
decimal degrees, the
voltage measurement of our sources, in units of $[V]^{2}$ (see Eq.(5)),
and the Local Sidereal Time ($lst$) (in decimal hours)  
and the Julian Date ($jd$) (in decimal days) at the time of the measurement.

The developed codes that controlled the telescope for the Moon, Sun and
point sources, storing the data as it performed the pointing at
desired intervals, can be found in my \emph{Github} account
\href{https://github.com/leosattler/ugradio/tree/master/lab_interf}{here}. 

\section{Point Sources}

We analyze measurements for three point sources: Orion Nebula, Virgo A
and M17. As outlined before, the output of a measurement is the product
of the intensities seen by both antennas, as in Eq.(6). Using
trigonometry identities, we can rewrite Eq.(6) into another meaningful
equation:

\begin{equation}
F(h_{s})=Acos\left[2\pi \left(\frac{B}{\lambda}cos(\delta)\right)sin(h_{s}) \right]
- Bsin \left[2\pi \left(\frac{B}{\lambda}cos(\delta)\right)sin(h_{s})\right],
\label{meanin}
\end{equation} 
where $A$ and $B$ are constants that replace the values $cos(2\pi \nu
\tau_{c})$ and $sin(2\pi \nu \tau_{c})$, respectively. Also, the $\nu$
from the previous equation was written as $\lambda$ via the relation $c =
\lambda \nu$. This is still the
fringe pattern, but now presented only with values that can be directly
measured from our interferometer configuration and source's position in
the sky. Our objective here is to perform a least square in our data and
determine the A and B values. For so, we write into the arguments of the
$cos$ and $sin$ the follwing value:

\begin{equation}
C = 2\pi\left(\frac{B}{\lambda}cos(\delta)\right),
\label{eq:1}
\end{equation}
also known as the \emph{fringe frequency} $f_{f}$ for our point
sources. 

To proceed with the least square fitting, we outline:

$\cdot$ Our values for $F(h_{s})$ are stored in our data as $volts$.

$\cdot$ We can get the corresponding values for $h_{s}$ from our source data by
subtracting its $lst$ values at each 
measurement by the $ra$ of this
source.

$\cdot$ Relevant physical values are:

\begin{table}[H]
\center
\begin{tabular}{|c|c|c|}
\hline
\multicolumn{3}{|c|}{Information Table}\\
\hline
Targets   & $ra$ & $dec$ \\ 
\hline
Orion & $5^{h}34^{m}17.3s$ & $-05^{\circ}23'28''$\\ 
Virgo A & $12^{h}30^{m}49.423s$ & $+12^{\circ}23'28.04''$\\ 
M17 & $18^{h}20^{m}26s$ & $-16^{\circ}10.6'$\\ 
\hline
\multicolumn{3}{|c|}{$B = 10 \ meters$}\\
\hline
\multicolumn{3}{|c|}{$\lambda = 0.028 \ meters$}\\
\hline
\end{tabular}
\caption{Relevant quantities to perform the least square fit.} 
\label{tab:1}
\end{table}

\subsection{Managing Data}

Our measurements are the products of measured intesities from the two
antennas. Besides astronomical effects, such as small brightness from
the sources, problems related to the instrument require a previous
manipulation of our data before any analysis. First of all, not all
intensities not corresponding to the source are zeroed out, so our data
comes with a lot of noise. The electric equipment ends up adding some
information as well, such as the DC offset. 

Some problems are due to the tracking/homing procedure. Our
tracking might not be optimal, so our measurements are not preciselly
what we theoretically expect. Also, the homing give spikes of minima
though the data, so we start our analysis by getting rid of them.

Fig.(\ref{m17}) shows the data before and after the manipulation. The top plot
is the data for Orion with a fitting curve in red. This function is
evaluated using a window and finding the average of values inside this
window and storing the result in the center. By subtracting the measured
volts by this curve we end up with a smoother data, much more reliable
to work with. Also, the spikes from the top plot were eliminated since
they are not actual measurements, but a result of the homing process.
 
This was done for the three point sources, and also for the Sun and the
moon. Now we may proceed to the least square using this smoothed data. 

\begin{figure}[H]
\center
\includegraphics[scale=0.37]{M_17_manip.png}
\caption {Data manipulation for M17. The red curve on the top plot is
  done by finding averages of a window across the data. The bottom plot
  is the measured data minus the red curve, giving a smoothed function.} 
\label{m17}
\end{figure}

\subsection{Least Square for Point Sources}

Given the values of Table 1 and the information from our data, our least
square is done to solve for the
coefficients $A$ and $B$. But they are constants that tell about the
amplitude of our measurements, and our goal here is to find quantities
that we can compare to previously measured values ($B$, for example). So
we proceed as follows:

$\mathbf{1.}$ We make a 'guess' for $C$ (Eq.(13)) based on our
information from Table 1, and create an array around this value.

$\mathbf{2.}$ for each element inside this array, we will evaluate an
$A$ and $B$ in order to have a fit and calculate the residuals.

$\mathbf{3.}$ By plotting the square of the residuals against our array
of $C's$, we can find the one that minimizes the curve and assume this is
our best $C$ representation.

With this value that minimizes the squared residuals, $C_{min}$, we get
back to Eq.(13) and evaluate the baseline separation $B$. This is the
procedure done for each source, and the resulting plots are presented next.

On Fig.(3) one least square fit for Orion is presented (in red). The
bottom plot is a zoom that shows how is the behavior of our least square
fit compared to the data. Although this
result seems not to be actually fitting the measurements, since the data
is noisy and hard to describe, not necessarily this fit will not be able to
statistically give informations about our data. In fact we move on with
this fit and calculate the residuals.

Fig.(4) is a graphical repesentation of the method for evaluation of
$C_{min}$. Plotted on the $y \ axis$ we have the square of the residuals
for the performed least squares of Orion, done for each value of $C$
(plotted on the $x \ axis$). When the $y$ value reaches a minimum, it
means that the function evaluated has the closest value to our
data. With the $C$ corresponding to this minimum, we refer back to
Eq.(13) and, assuming known $cos(\delta)$ and known $\lambda$, we solve
for the corresponding baseline $B$. 

The least square plot for the other point sources are at my
\emph{Github} account page, on the folder called
\emph{lab\_interf}. On Table 2 the evaluated $C_{min}$ and baselines $B$
are presented.

\begin{table}[H]
\center
\begin{tabular}{|c|c|c|}
\hline
\multicolumn{3}{|c|}{Final Evaluated Quantities}\\
\hline
Target    & $C_{min}$ &  Baseline ($B$)   \\ 
\hline
Orion &  $352.5695$ & $9.9156$ \\ 
Virgo A & $320.8382$  & $9.1973$ \\
M17 & $341.0037$ & $9.9417$ \\ 
\hline
\end{tabular}
\caption{Evaluated values for each point source.} 
\label{tab:1}
\end{table}


\begin{figure}[H]
\center
\includegraphics[scale=0.37]{Orion_lq.png}
\caption {Least Square fit for Orion. The bottom plot is a zoom of the
  top one, to evidentiate the fitting pattern.} 
\label{m17}
\end{figure}

\begin{figure}[H]
\center
\includegraphics[scale=0.37]{sqr_res_orion.png}
\caption {Plot of the squared residuals for each value of C. The green
  line spot the value where the squaerd residuals reach a minima,
  corresponding to a good value of C (that best describes the data).} 
\label{m17}
\end{figure}

\section{Sun and Moon}

For the Sun and the Moon we get some particularities that changes our
process of taking and analysing the data. First of all, as discussed
before, their Right Ascention and Declination change within a day, so
their tracking function is different from the one used to the point
sources, since keep updating their ra and dec from time to time. 

About the data analysis, these objects are in fact resolved in the sky,
which means that their angular size are appreciable hence are not perceived
as a dot by our interferometer beam. What it does to the data is that
our fringe pattern will have an amplitude variation. As
the Sun and the Moon move through the sky, the contribution of both
antennas will eventually get cancelled. An equation that describes this
effect is 

\begin{equation}
R(h_{s})=F(h_{s}) \times \int I(\Delta h)cos(2 \pi f_{f} \Delta h)
d\Delta h,
\label{mod}
\end{equation} 
where we see from Eq.(\ref{meanin}) that the first factor is the point
source fringe, and the second is a modulator. $I$ is the intensity of
the source, $\Delta h = h - h_{s}$ is the hour angle $h$ relative to the
$h_{s}$ of the source center and $f_{f}$ is the fringe frequency, a very
important quantity to be discussed on the followinf sections. 
This second factor had to be taken into account for
extended sources, while for objects as Virgo, M 17 and Orion, $\Delta h
=0$ and we had $R(h_{s}) = F(h_{s})$ for them. 

\subsection{Least Square for Sun and Moon}

Here we investigate the modulation of our data. Zeros of this modulation
should occur when integral number of fringe periods occur inside the
source width, where there will be an equal contribution of negative and
positive values of the fringe, leading to a net integral equals to zero.

The next plot shows the Sun and Moon data, presenting the expected
oscilation, specially for the Sun (top plot).  

\begin{figure}[H]
\center
\includegraphics[scale=0.36]{sun_moon.png}
\caption {Plot of Sun (top) and Moon (bottom). Their shape are modulated
by the factor presented on Eq.(\ref{mod}), and compared to the point
source data it indeed shows a variation of its amplitude.} 
\label{Fig:3}
\end{figure}

We now apply a least square to the Eq.(\ref{mod}) to determine the
fringe modulator function. Initially we are suppose to find an optimal
value for $\phi$ in the follwing equation:

\begin{equation}
F(h_{s}) = cos(2\pi \frac{B}{\lambda}  cos(\delta)  sin(ha) + \phi),
\label{modd}
\end{equation} 

where $\phi$ is the phase offset of the fringe source, and $F(h_{S})$
describes the amplitude of the data. This equation is achieved via
trigonometry functions from the Eq.(\ref{mod}).

We now choose a window of values from our data that contains an apparent
drop of amplitude being exposed (see Fig.(\ref{sun_win}) and Fig.(\ref{moon_win})). We solve a least square by fitting a
parabola to the data from this window, and, similarlly to what was done
to calculate $C_{min}$, we get an optimal value for $\phi$ from a
range of guesses, by storing the sum of the squared residuals for
different least squares with different $\phi's$, and later with a plot
we spot the $\phi_{min}$ that minimizes the residuals.

Our parabola for the least square is like the following:

\begin{equation}
X =
 \begin{pmatrix}
  F(h_{s}) & F(h_{s}) \times ha & F(h_{s}) \times ha^{2} \\
  F(h_{s}) & F(h_{s}) \times ha & F(h_{s}) \times ha^{2}\\  
  F(h_{s}) & F(h_{s}) \times ha & F(h_{s}) \times ha^{2}
 \end{pmatrix},
\end{equation}
based on the hour angle measurements of our object (here, either the Sun
or the Moon). After finding the $\phi_{min}$, we proceed to a last least
square fit and get a shape of what would best describe the parabola of
the window. This optimized shape has the information of one of the zeros
across our data, described by the fringe frequency,

\begin{equation}
f_{f} = \left(\frac{B}{\lambda}cos(\delta)\right)cos(h_{s}).
\label{f_f}
\end{equation} 
The index of the minimum value of this parabola fit, presented on the bottom
of Fig.6 for the Sun and of Fig.7 for the Moon, will give the values we use to calculate the
previous quantity $f_{f}$ for each of them, since the declination and the hour angle
($h_{s} = lst - \alpha$) corresponding to this index indeed represent
a point where our data reaches a minimum.

\begin{figure}[H]
\center
\includegraphics[scale=0.37]{sun_win.png}
\caption {Figure showing the selected window position (top), the
  envelope fitting (middle), and the final least square fitting (bottom)
  for the Sun.} 
\label{sun_win}
\end{figure}

\begin{figure}[H]
\center
\includegraphics[scale=0.37]{moon_win.png}
\caption {Figure showing the selected window position (top), the
  envelope fitting (middle), and the final least square fitting (bottom)
  for the Moon.} 
\label{moon_win}
\end{figure}

\subsection{The Modulating Function and Results} 

The fringe modulator of Eq.(\ref{mod}) can be evaluated by a numerical
solution of an integral
that only depends on our fringe frequency $f_{f}$ and on the Radius of
the source $R$. This integral solution is an approximation of a Bessel Function,
which is the result of a Fourier Transform of a circle. This is expected
since \emph{the modulating function is a Fourier Transform of the source intensity
distribution on the sky}. The integral is defined as (see handout for
further details): 

\begin{equation}
MF_{theory} \approx \frac{R}{N} \sum
_{n=-N}^{n=+N}\left[1-\left(\frac{n}{N}\right)\right]^{1/2}cos\left(\frac{2\pi
    f_{f}Rn}{N}\right).
\label{M_f}
\end{equation} 
Since this integral only depends on the product $Rf_{f}$, the zero
occurances will give important information about the source
structure, since they occur, as commented previously, when the whole
disk of the Sun (or of the Moon) are providing a minimum on their
data.  

The next plot shows the approximation of a Bessel function for the
Sun. At each zero crossing we can evaluate $R$, since it will correspond
to a zero at Eq.(\ref{M_f}), hence the argument of the $cos$ must be
one, and since all values inside this function are constant, the
variation is given by our product $Rf_{f}$ that can be solved for $R$
since we know $f_{f}$ from Eq.(\ref{f_f}). One can compare the zero
positions on this graph with the minimum positions along the Sun's data on
the top of Fig.'s (5) or (6), and try to guess which one corresponds to which.

\begin{figure}[H]
\center
\includegraphics[scale=0.37]{bess_sun.png}
\caption {Sun's modulating function. Each zero crossing is a minima on
  the Sun's data, corresponding to a number tha makes possible the
  evaluation of the Radius of the sun for a known $f_{f}$.} 
\label{Fig:3}
\end{figure}

The next Table presents the evaluated svalues for the Radius of the Sun and the Moon:

\begin{table}[H]
\center
\begin{tabular}{|c|c|c|c|}
\hline
\multicolumn{4}{|c|}{Final Evaluated Quantities}\\
\hline
Target    & $f_{f}$ & $R$ in Radians & $R$ in arcmin \\ 
\hline
Sun & $115.786669$ & $0.004318$ &  $14.844190$\\ 
Moon & $356.302406$  & $0.004209$ & $14.469476$ \\
\hline
\end{tabular}
\caption{Results for the Sun and Moon.} 
\label{tab:1}
\end{table}

\section{Conclusion}

In this lab we learned the working of an interferometer by dealing with
its physical properties, and via data analysis we recovered informations
about the instrument and source's geometry. And, in order to perform
data aquisition, we went through Coordinate Systems, so that we were
able to use the \emph{radiolab} module and control the interferometer
antennas. These procedures were correctly done, based on the reliability
of our data and the decent values for the evaluated quantities,
evidenciating the success of the developed codes.

For the point sources, the values for Virgo A are much different from
the expected, $B = 10 \ meters$, reaching $91.97\%$ of the expected
value, while for Orion we achieved $99.15\%$ and for M 17 $99.41\%$ as
estimatives of the nominal value for the baseline. It can be justified
by the fact that, the greater the brighness of the source, more
information about it we collect, hence more reliable are our
measurements. For the three of them, Virgo A is the one with the smallest
value of \emph{Flux Density} ($\sim 34 \ Jy$), while Orion, our second best
estimative, has ($\sim 340 \ Jy$), and M17, our best result, has ($\sim
500 \ Jy$).

Also, these results indicate that, although the fitting from the least
square procedure was not ideal (see Fig.(3)), it indeed was able to
provide statistical informations that allowed us to find reasonable
values.

The extended sources had values around the minimum expected. For the
Sun, the found value is around $93.95\%$ of the minimum angular size of
the Sun in the sky. For the Moon, it is $98.76\%$ from its minimum
size, what is a good estimative. For the Sun, the chosen window was the
first zero of our data, and for the Moom it was the second one. 

\end{document}


