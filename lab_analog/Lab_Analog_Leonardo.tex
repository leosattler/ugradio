\documentclass{article}
\usepackage{amsmath,amsfonts}
\usepackage{graphicx}
\usepackage[margin=1.0in,letterpaper]{geometry}
\usepackage{float}
\usepackage{subfigure}

% are used to denote commments, everything after it on this line is a comment and while not be processed 

\begin{document}

\title{Analog Circuit}
\author{Leonardo Sattler Cassara, e-mail:leosattler@berkeley.com}
\date{February 10, 2014}
\maketitle

\begin{abstract}
In this lab we work on the physics of an analog circuit, by building a
radio demodulator and an amplifier. We understand the physics behind the transmission of radiowaves and signal transmissions, by developing our own circuits using breadboards and analysing signals with an osciloscope. The design and construction were made in group by the following students: Isaac Domagalski, David Galbraith and Leonardo Sattler. At last, we study sources of noise related to electric circuits and develop Python codes (individually) to enhance its comprehension. 

\end{abstract}

\section{Introduction}

Radio players deal with several electronic components that, added
together on a certain order, will generate different features with huge
physical meaning behind them. With the purpose of understanding the
theory behind it, we describe separately the working of these
components, and simulate their behavior with generated graphs. The plots
are a way of picturing what happens inside these circuits.

To play an FM signal is our main goal, but transmission, enhancement and
lost of these signals are discussed, so as their impact over the system
and how one generally deals with it. This is done with an statistical
approach over some of these problems with developed codes
in $Python$, and the results are physically interpreted.

\section{FM Demodulator} 

In order to understand what is an analog circuit we built a radio receiver on a breadboard, by using electronic devices provided by the Undergraduate Lab at UC Berkeley. The basic concept of this device is, by the receiving
of a radio signal it will, first, translate this signal into audio information and than enhance it by an electrical filter (see Fig.1). The scheme presented was used to reproduce FM (Frequency Modulated) waves from a source located in the lab. 

\begin{figure}[H]
\center
\includegraphics[scale=0.55]{cic-1.png}
\caption {Sketch of the FM demodulator circuit built for this lab.} 
\label{Fig:3}
\end{figure}

\begin{table}[H]
\center
\begin{tabular}{|c|c|}
\hline
Electric Component    &   Value    \\ 
\hline
$R_{1}$ & $33 \ \Omega$ \\ 
\hline
$R_{2}$ & $150 \ \Omega$ \\ 
\hline
$R_{3}$ & $150 \ \Omega$ \\ 
\hline
$R_{4}$ & $150 \ \Omega$ \\ 
\hline
$C_{1}$ & $22000 \ pF$ \\ 
\hline
$C_{2}$ & $10000 \ pF$ \\ 
\hline
$C_{3}$ & $10000 \ pF$ \\ 
\hline
$C_{4}$ & $10000 \ pF$ \\ 
\hline
$L$ & $1 \ \mu H$ \\ 
\hline
\end{tabular}
\caption{Components specifications. The values were calculated by looking at reference tables for Resistors and Capacitors.} 
\label{tab:1}
\end{table}

The very first part is an antenna that receives the signal from the FM source.
The oscilating signal produces a flow of electrons that will run through the circuit 
as an alternating current $AC$. The only external power here is the $+5V$ battery providing 
direct curent $DC$, that will not be over the whole circuit. 

The reason is because the capacitors $C_{2}$ and $C_{4}$ have their own job of blocking direct
current. Made of parallel plates that, once in equilibrium after
charged, no flow of electrons will pass by them. 
From a practical point of view it devides the circuit, and the
$+5V$ voltage source won't act on the electrical components after
$C_{2}$ and $C_{4}$. The other elements act together creating electrical
devices that will determine the working of the whole circuit. They are
presented and described on the next sections.

\subsection{RLC Filter}

By using an antenna, it first captures the signal. After passing by
$R_{1}$, it reaches a device that will translate the variable frequency
wave into an Amplitude Modulated wave, which is a signal with a variable
amplitude. It generates an \emph{envelope} over this signal, now called \emph{carrier} (see Fig.\ref{carrier-envelope}). This device, together with $R_{1}$, is known as RLC filter, and the other components are a \emph{Capacitor} and an \emph{Indutor} ($L$ on Fig.1). 

\begin{figure}[H]
\center
\includegraphics[scale=0.40]{envelope-carrier.png}
\caption {Figure showing the wave after passing by the RLC filter. An envelope over the amplitude modulated signal, the carrier.} 
\label{carrier-envelope}
\end{figure}

Series RLC circuits give minimum impedance at the so called Resonance Frequency. Parallel RLC (also known as 'tank', used here) give maximum \emph{impedance} (technical name for the resistence
created by the circuit components) at their resonant frequency. It is used to short the entrance signal based on its frenquency. Our receiving signal is a faked FM station at $1.045 \text{ }MHz$. The resonant frequency is calculated with the following formula:

\begin{equation}
f_{0}=\frac{1}{2 \pi \sqrt{LC}},
\label{eq:1}
\end{equation}
and by looking at Table 1 we have $f_{0}=1.073 \text{ }MHz$. 

The basic
physics behind this device is that Indutor and Capacitor store
electromagnetic energy by themselves by the passing of an electric
current. They store but also release, and this keep going as an
oscilating pendulum, but the energy fades with time. This will
generate a change in amplitude for the signal (recall that $Energy \text{
  }\alpha \text{ }V^{2}$), and its minimum possible value will be
the one closer to $f_{0}$. It is clearly shown by the equation

\begin{equation}
Z(\omega) = -j\frac{1}{C}\frac{\omega}{\omega^{2} - \omega^{2}_{0}},
\label{eq:1}
\end{equation}
where $Z$ is the letter for impedance, $\omega=2 \pi f$ is the incoming frequency and $\omega_{0}=2 \pi f_{0}$ is the natural frequency of oscilation from Eq.1. It is straightforward to realize the growth of $Z$ as $\omega \rightarrow \omega_{0}$.

Our working FM station produces waves at
$1.045 \text{ }MHz$. We built our circuit with $R$, $C$ and
$L$ that could match something close but not at $1.045 \text{
}MHz$, creating high \emph{resistence} for these frequencies and
allowing $f_{0}$ to survive (almost) alone after the filter. Fig.\ref{LC} shows the
behavior of the impedance due to the $LC$ values chosen, and
the maximum impedance happens on the $f_{max}$ specified on the graph,
marked with the green dashed line. 

\begin{figure}[H]
\center
\includegraphics[scale=0.28]{1-7-1.png}
\caption {Impedance response to a given frequency input. The value of $f_{max}$ corresponds to the one of maximum impedance, and as expected matchs with $f_{0}$.} 
\label{LC}
\end{figure}

\subsection{Envelope Detector}

\subsubsection{Biasing and Diode}

Once we have a well behaved Amplitude Modulated signal, we use a \emph{Diode} ($D$ on Fig.1) to remove the negative part of the incoming wave, composed by the \emph{carrier} and the \emph{envelope} (see Fig.2). But first we apply a Voltage source ($+ \text{ }5V$) to power up the circuit. It adds a step voltage to the oscilating signal, basing it, and at the same time gives more current flow for the diode. The Resistors $R_{2}$ and $R_{3}$ act as voltage dividers, lowering its value through the circuit. 

The diode works by allowing the current flow through only one direction:
positive or negative (in our case positive). Since we have an alternating current on the circuit,
the diode will block any backing current and no negative value of the
oscilating wave will survive after that. This feature allows the capacitor only to charge, since no
signal with opposte value will pass by it. It charges after each oscilation of the carrier wave, 
never discharging until it reaches its maxima, when it will keep following its peaks. 
The result is a wave profile as
the one of the the envelope but with no negative part and no carrier,
taken out by its association with a Resistor, presented
next, creating a device that is presented next. 

\subsubsection{RC Filter}

In order to clean the internal waves of the envelope a RC Filter is
used with the diode. They together are known as Envelope
Detector, since the output will be only the positive part of the
envelope originally travelling with a carrier. 

The RC configuration used above is called \emph{Low Pass Filter}. It
literally allows only small frequencies to survive, hence the
desapearing internal waves (with high frequency in contrast to its
envelope). As we can see on Fig.\ref{lastfilter}, the so-called cutoff
frequency is the one at which the sgnal is atenuated by $3dB$, happening
at 

\begin{equation}
\omega_{3dB}=\frac{1}{RC},
\label{eq:1}
\end{equation}
defining the bandwith that survives due to this device. On the graph we
see that the bandwith is composed by frequencies smaller than
$\omega_{3dB}=106103.295$, placed on the left part before the vertical
green dashed line.

\begin{figure}[H]
\center
\includegraphics[scale=0.31]{1-7-2.png}
\caption {Graph showing the impedance response due to the input
  frequency. The vertical green dashed line marks the \emph{cutoff
   frequency}, and the horizontal one the $3dB$ of atenuation on the
 input signal, corresponding to a $70.7\%$ of lost, defining the bandpass.}
\label{lastfilter}
\end{figure}

The input FM signal, after turned into an AM wave with an amplitude and
frequency very well defined, reaches the RC filter for an extraction of
what really is the audio signal. Now it is able to be the input of an
amplifier before reaching our ears. 

\section{Amplifier}

Amplifiers are devices that enlarge the power of a given input, by a
certain ratio called \emph{gain}, basically by the usage of
\emph{Transistors} in association with resistors (See
Fig.\ref{circuit2}). The one built here is known as Emitter-Follower.

\begin{figure}[H]
\center
\includegraphics[scale=0.53]{circ-2.png}
\caption {Sketch of the circuit of an Amplifier built for this lab.} 
\label{circuit2}
\end{figure}

\begin{table}[H]
\center
\begin{tabular}{|c|c|}
\hline
Electric Component    &   Value    \\ 
\hline
$C_{1}$ & $1 \ \mu F$ \\ 
\hline
$C_{2}$ & $1 \ \mu F$ \\ 
\hline
$C_{3}$ & $1 \ \mu F$ \\ 
\hline
$R_{1}$ & $3900 \ \Omega$ \\ 
\hline
$R_{2}$ & $470 \ \Omega$ \\ 
\hline
$R_{C}$ & $510 \ \Omega$ \\ 
\hline
$R_{E}$ & $220 \ \Omega$ \\ 
\hline
\end{tabular}
\caption{Components specifications for the Amplifier circuit.} 
\label{tab:1}
\end{table}

Again, $C_{2}$ is isolating the circuit from the DC current
provided by the external source. $C_{1}$ also isolates this current from the
RM Demodulator of the last section, that will provide the signal to be
amplified. $C_{2}$ separates the amplifier from the speaker that will be
playing the output of this circuit. The other components (including
$C_{1}$, that plays another role here) are described
in the following sections.

\subsection{Transistors}

Transistors are composed by 2 diodes that work under certain
conditions. On the next diagram, a $BjT$ (\emph{Bipolar junction Transistors}) is presented with its terminals, $B$, $C$, and $E$. They are the Base, Collector and Emitter. The voltage difference between base and emitter ($V_{BE} \approx 0.6V$) is constante, so if $V_{in}$ drops, so does $V_{E}$, and will rise if $V_{in}$ incrieses:

\begin{equation}
V_{out}=V_{in}-V_{BE}.
\label{V}
\end{equation}
The same between collector and emitter ($V_{CE} \approx 0.2V$). This is guaranteed by the collector, that allows the current to pass given a DC source ($+31.5 \text{ }V$). Both collector and emitter will cease the current and stop working if these conditions are not satisfied.

\subsubsection{Biasing the Circuit}

And how does the amplfication work? It is related to the current and the resistence at the terminals of the transistor. Once respected the relation of the previous equation, we have for the currents:

\begin{equation}
I_{E}=I_{C}+I_{B}, \text{ or }I_{E}=(\beta +1)I_{B},
\label{I}
\end{equation}
where $\beta=I_{C}/I_{B}=60$, a nominal value of the used
transistor. The key understanding is that, with a \emph{biasing source} as the
$+31.5V$ used here, the current flow will be respecting the rules of
Eq.'s(\ref{V}) and (\ref{I}) and won't care for the impedance of the
load, in our case a speaker. It means that, even with a small current on
the input $I_{B}$, provided by $V_{in}$, the current $I_{E}$ at
$V_{out}$ will be a lot larger, in such a way that $I_{B}$ is negligible
on Eq.(\ref{I}). 

What does it mean in terms of power? Having a higher current flow means
more energy ($Energy \ \alpha \ I^{2}$), so this and not a voltage gain is
the responsible for amplifying the signal (as one can see from
Eq.({\ref{V}), $V_{in}-V_{out}=V_{BE}=0.6$ is not a considerable
  value).   

Since, as discussed in the begining of Section 2.1, transistors only work if $V_{BE}$ is
over a threshold, the values of $R_{1}$ and $R_{2}$ are chosen to be high in order to avoid too much current from
the biasing voltage source to flow toward the base terminal, and
guarantee its flow through collector and emitter. In addition to that,
once specified the value of the biasing voltage source ($+31.5V$) and the
impedance value of $Z_{C}=R_{C}$, $I_{C}$ is know from \emph{Ohm's Law}
($V=RI$), and so is $I_{E}$ (see Eq.(\ref{I}), and recall that $I_{B}$
is too small compared to $I_{C}$). For a given $Z_{E}$ it
also means a known $V_{E}$.

With the quantities calculated above, one can choose
the right $R_{1}$ and $R_{2}$ to work as voltage deviders, and drop the
voltage from the biasing source until the desired value for the terminal
$V_{B}$. This will give the transistor the minimum value
$V_{BE}=V_{B}-V_{E}$ to satisfy its working conditions, even if there's no voltage
input $V_{in}$. 

\subsubsection{Gain}

To measure how much is the amplification, one can calculate the gain
given by this configuration. It is the ratio of the impedance on the
collector terminal and of the emitter termial. The impedance $Z_{C}$ on
the collector is simply $R_{C}$. On the emitter, $Z_{E}$ it is the impedance of a
resistor ($R_{E}$) in parallel association with a capacitor
($C_{3}$). As a result:

\begin{equation}
g = \frac{Z_{C}}{Z_{E}}=\frac{R_C}{R_E}\sqrt{(1+R^{2}_{E}\omega^{2}C_{3}^{2})},
\label{g}
\end{equation}
and one can choose $R_{E}$, $R_C$ and $C_{3}$ in order to achieve a 
desired value for $g$ (see next graph).

\begin{figure}[H]
\center
\includegraphics[scale=0.32]{gain.png}
\caption {Amplifier response related to the frequecy input. The gain is
  higher for higher frquencies.} 
\label{gain}
\end{figure}

As seen on Fig.(\ref{gain}), the gain is higher for higher
frequencies. It is like this because the calculus of our gain, which
evolves the impedances at the collector and emitter (see Eq.(\ref{g})), has to deal with the
capacitor at the emitter's terminal, and it has a variable impedance that
depends on the receiving frequency. 

\subsection{Another RC Filter}

This result for the gain is important because, despite all that has been
said about $R_{1}$ and $R_{2}$ so far, together with
$C_{1}$ they compose what is called a \emph{High Pass Filter}, located
at the very begining of our circuit diagram.

The filter described on Subsection 2.2.2 was a
capacitor following a resistor creating a Low Pass. When the opposite is
true, and a capacitor comes before the resistence, they work in such a
way to have a maximum impedance for low frequencies. The cutoff
frequency is again 

\begin{equation}
\omega_{3dB}=\frac{1}{RC}, 
\label{3dB2}
\end{equation}
but due to the different configuration the result is as shown in
Fig.(\ref{high}) (it is interesting to compare with Fig.(4)).

This is totally reasonable since we are designing an audio amplifier, so
we want it to operate frequencies on the range $20 \ Hz$ to $20k \
Hz$. This is, as seen on Fig.(\ref{high}) ($x$ axis in log scale), \
what this filter will provide. The bandwith that survives after it
belongs to the range of $\approx \ 400Hz$ to higher than $20000 Hz$.

\begin{figure}[H]
\center
\includegraphics[scale=0.32]{high-pass.png}
\caption {Amplifier response related to the frequecy input. The gain is
  higher for higher frquencies.} 
\label{high}
\end{figure}

\subsection{Impedances}

In the overall, powering a circuit is a matter of giving more current than the input
voltage $V_{in}$ provides, reaching the load with a
factor called gain that increases it. Another way to understand this
phenomena is realizing that the input has an impedance different from the
output. For example, looking into the base one sees an \emph{input impedance} of
$Z_{in}=R_{in} \beta$, while looking into the emitter sees an \emph{output impedance}
smaller than that, equals to $Z_{out}=R_{out}/\beta$. In this case
$R_{in}$ is any impedance connected to the emitter, and $R_{out}$ is
whatever impedance connected to the base. 

It tells the upcoming signal $V_{in}$ that the circuit has an impedance
much larger than the one seen by the load, and that is why even with an
small input current the voltage on the output is powered and we see an
amplification of the original signal. 

But impedances are relevant not only for power gain, but also to avoid
its lost. In a \emph{Transmission Line}, a wire that propagates the
signal from a source to a load, impedance matching is a key factor. It
is true because the signal can reflect either with  positive or negative
sign when finding a different impedance along its way. The reflection will
be positive if there's a load with higher resistence than the one of the
cable, and negative if the load has a lower resistence. Reflections are
not desired since it causes interference with the upcoming waves, so to
avoid this we seek for an impedance matching between source, cable and
load.

If we are to transmit the output of the amplifier to a speaker with a
wire, we should look for the impedance of both. To calculate the
impedance of a transmission line there's the following equation,

\begin{equation}
Z_{0}=\sqrt{\frac{L}{C}},
\label{g}
\end{equation}
which is interesting since does not depend on the cable's length. The
impedance of the speaker, as of other loads, is usually known. The
objective is than look for a way to match the impedance of the cable
with both source and load. But maximizing power transfer is not the same
as maximizing power efficiency ($\eta$):
  
\begin{equation}
\eta=\frac{R_{oad}}{R_{source}+R_{load}}=\frac{1}{1+\frac{R_{source}}{R_{load}}},
\label{eta}
\end{equation}
showing that when $R_{laod}=R_{source}$ we have $\eta=0.5$.

However, impedance mismatching can lead to excessive power use,
distortion and noise problems. The most serious problems occur when the
impedance of the load is too low, 
requiring too much power from the source to drive the load with the
signal. So the importance of matching impedances goes beyond just
avoiding reflections, but as showed in Eq.(\ref{eta}) it does not mean a
transmission of the total power from our amplifier.

\section{Noise}

Well built audio circuits will provide high audio fidelity. But the
sound quality can be compromised by several factors, and the
\emph{Noise} is one of them. The Johnson-Nyquist noise is characterized as the 
thermal noise generated by the random motions of electrons on a resistor. This motion
generates a resultant flow that is interpreted as a current by the
circuit and a voltage appears. 

The equation describing this phenomena is 
\begin{equation}
V^{2}=4K_{B}TBR,
\label{v}
\end{equation}
where $K_{B}$ is the Boltzmann’s constant, $T$ the temperature of the
resistor, $B$ is the frequency bandwidth of the circuit
and $R$ is the resistance of the resistor. The Power signal ($P_{sig}$)
generated is easily derived, since $P_{sig}=V^{2}/R$:
\begin{equation}
P_{sig}=4K_{B}TB.
\label{p}
\end{equation}

This power will propagate through the circuit and be a source of noise
for it. This is also known as the \emph{White Noise} of the circuit, and
since the random motion of electrons is proportional to the temperature on the resistor,
the higher it is more voltage will appear. Besides that,
Eq.(\ref{p}) also tells that larger bandwith will increase the noise. So
a way to reduce it is by either cooling the resistor or working with a
smaller bandwith on our systems.

\subsection{Fourrier Transform}

\emph{Fourrier Transform} is a mathematical tool used to convert any
signal from time domain to frequency domain and vice versa.
If we analyse the voltage (or current) flowing over time on the resistence generating the
noise and apply a Fourrier Transformation, we find a cutoff frequency
for which this thermal noise will cease (which high,
$\approx50 \ Ghz$). And the power noise as a
function of frequency ($P_{sig}(\omega)$) appears to be \emph{flat}, meaning that
we usually receive the same amount of noise for a big range of
frequency, untill it starts cutting off (around the cutoff frequency).

Since our circuits usually operate under this cutoff frequency, we
associate our noise to a determined bandwith,
\begin{equation}
P_{sig}=K_{B}TB,
\label{last}
\end{equation}
and $B$ is the range of frequency determined by our low pass and high
pass filters, for example. The drop of the $4$ from Eq.(\ref{p}) is
because a source only maximize its power transfer to a load when we have an impedance
matching, as discussed on Section 3.3. In order to have this power noise
runing over the circuit, we can think the resistor generating it as a
source, and the impedance of the circuit as a load. From Eq.(\ref{eta}),
this impedance matching will provide a factor of $0.5$ over the voltage
output and hence a factor of $0.25$ for $P_{sig}$.

\subsection{Random Electrons}

The thermal noise is a source of randomic error that acts over
electrical circuits. But within the time it shows an amplitude
distribution that corresponds to a  \emph{Gaussian}, what can be
ilustrated with the \emph{Central Limit Theorem}. 

This theorem says that, on the limit of a large quantity of 
sets of some variables from any random distribution, the ensemble of
these sets emerges on a Gaussian distribution. To prove so, a code 
was developed to apply the Central Limit Theorem 
to a distribution created with a randomic range of integers 
by the usage of the \emph{numpy.random} function in \emph{Python}.
The result is shown on Fig.(8).

\begin{figure}[ht]

  \centering

  \begin{tabular}{cc}

    % Requires \usepackage{graphicx}

    \includegraphics[width=71mm]{hist1.png}&

    \includegraphics[width=71mm]{hist2.png}\\

    \includegraphics[width=71mm]{hist3.png}&

    \includegraphics[width=71mm]{hist4.png}\\

  \end{tabular}

  \label{hists}\caption{These histograms show how the means
    of sets with a given sample size, when added over and over n times, 
  will approach a Gaussian Distribution, showing that the
  Johnson-Nyquist noise will act with a gaussian profile over the
  system. Histogram 1, with $n=200$, has a randomic shape that is not
  much of a gaussian. As $n$ grows, untill $n=15000$ on
  Histogram 4, we see the gaussian profile taking form over the mean occurrences.}

\end{figure}

Also, for any random distribution, a larger number of samples used to
calculate the mean will result on a smaller standard deviation for the
converged gaussian noise. This is islustrated by Fig.(\ref{std}), where
a curve of the form 

\begin{equation}
\sigma = \frac{1}{\sqrt{N}}
\label{p}
\end{equation}
was used to fit the data, $N$ being the number of samples and $\sigma$
the standard deviation of these sample means. 

Physically speaking, the averaged
velocities of the electrons inside a resistor will add up $n$ times until
converge to a resultant value, that is different from
zero and will be described by a gaussian.  
Besides that, the larger is the $N$ number of electrons randomly
contributing to this net velocity, smaller will be the standard
deviation of this gaussian.    

A possible way to experimentally check the
Johnson-Nyquist noise is to plug a cable on a resitor, properly heat it
and check the noise appearing on an osciloscope or powermeter. 
Since this signal is supposed to be small, amplifiers are required to
enhance it. However, some points as impedance matching and the signal
noise of the own amplifiers must be observed, since it must be
signficantly small compared to the resistor's noise and a person might want
to use more than one to get more gain. Another problem is
choosing a filter that will supply the correct bandwith for the used
amplifiers, filtering their signal before reaching the osciloscope.

If the result is not an increase of the noisy signal as you plug your
resistor to end of the cable, something is amiss. Even if following
these conditions, one must recall that the \emph{Signal to Noise Ratio} is what
comes into play while trying to observe the resistor's noise above the
noise generated by your configuration. A deeper study over this can
drive more general explanations. For example, the information transmission, on
a system of a specified bandwith in the presence of a gaussian noise, 
will be the maximum according to the \emph{Shannon-Hartley
  Theorem}. These are relevant subjects to be understood while conducting this experiment.  

\begin{figure}[H]
\center
\includegraphics[scale=0.32]{std.png}
\caption {Standard Deviation of the mean of N random samples. Increasing
N, the standard deviation drops as describe by the red line: $\sigma \ \alpha
 \ 1/\sqrt{N}$.} 
\label{std}
\end{figure}

\section{Conclusion}

In this lab, the construction of circuits was very important to
understand the physical concepts behind electronic equipments, and this
was achieved by applying the knowladge of different associations between
indutors, capacitors, and resistors, also by understanding the working
of a diode and a transistor. As seen on the Fig.'s (3), (4), (6) and
(7), the theoretical plots were consistent with the experimental results
read on the oscilospe, when analysing the difference between input and
output signal over the buit devices.

We can firmly state that the specified values used on Tables 1 and 2
were propperly chosen, since both FM Demodulator and Amplifier were
giving the expected output on the osciloscopes. 
More than that, they were attached together and plugged into a
speaker, tunned to a FM station and the result was as expected: an
amplified audio signal (music!). By decoupling the
circuit of Fig.(5) from the one of Fig.(1), the volume was lower as
expected.

The Johnson-Nyquist noise profile over the circuit was well descibed by
the presented graphs, as a result of the developed code 
(sent together with this lab report), that provided the 
physical association between the Central Limit Theory and the thermal
noise. 

The experimental difficulties related to the proof of this phenomena
were presented, and relevant topics about the theory of signal
propagation and detection were cited, as starting points to 
overcome these difficulties.

\end{document}


